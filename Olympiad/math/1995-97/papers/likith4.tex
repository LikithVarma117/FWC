\documentclass{article}

\usepackage{amsmath}
\usepackage{amssymb}
\usepackage{fullpage}
\usepackage{graphicx}

\begin{document}

\noindent
\begin{minipage}[t]{0.48\textwidth}
    \vspace{0pt}
    \includegraphics[width=4cm]{logo1.jpg}
\end{minipage}
\hfill
\begin{minipage}[t]{0.48\textwidth}
    \vspace{0pt} %
    \raggedleft
    \textbf{Name:} Likith Varma\\
    \textbf{ID:} COMETFWC048\\
    \textbf{Date:} 03-02-26\\
\end{minipage}

\vspace{1.4cm}

\begin{center}
\textbf{36$^{\text{th}}$ International Mathematical Olympiad}\\
First Day -- Toronto -- July 19, 1995\\
Time Limit: $4\frac{1}{2}$ hours
\end{center}

\vspace{0.6cm}

\begin{enumerate}

\item
Let $A,B,C,D$ be four distinct points on a line, in that order.
The circles with diameters $AC$ and $BD$ intersect at $X$ and $Y$.
The line $XY$ meets $BC$ at $Z$.
Let $P$ be a point on the line $XY$ other than $Z$.
The line $CP$ intersects the circle with diameter $AC$ at $C$ and $M$,
and the line $BP$ intersects the circle with diameter $BD$ at $B$ and $N$.
Prove that the lines $AM$, $DN$, $XY$ are concurrent.

\item
Let $a,b,c$ be positive real numbers such that $abc=1$.
Prove that
\begin{align*}
\frac{1}{a^{3}(b+c)}
+ \frac{1}{b^{3}(c+a)}
+ \frac{1}{c^{3}(a+b)}
\ge \frac{3}{2}.
\end{align*}

\item
Determine all integers $n>3$ for which there exist $n$ points
$A_{1},\ldots,A_{n}$ in the plane, no three collinear,
and real numbers $r_{1},\ldots,r_{n}$ such that for
$1 \le i < j < k \le n$, the area of $\triangle A_iA_jA_k$
is $r_i + r_j + r_k$.

\end{enumerate}

\vspace{1cm}

\begin{center}
\textbf{36$^{\text{th}}$ International Mathematical Olympiad}\\
Second Day -- Toronto -- July 20, 1995\\
Time Limit: $4\frac{1}{2}$ hours
\end{center}

\vspace{0.6cm}

\begin{enumerate}

\item
Find the maximum value of $x_0$ for which there exists a sequence
$x_0,x_1,\ldots,x_{1995}$ of positive reals with $x_0=x_{1995}$,
such that for $i=1,\ldots,1995$,
\begin{align*}
x_{i-1} + \frac{2}{x_{i-1}}
= 2x_i + \frac{1}{x_i}.
\end{align*}

\item
Let $ABCDEF$ be a convex hexagon with
$AB = BC = CD$ and $DE = EF = FA$,
such that $\angle BCD = \angle EFA = \pi/3$.
Suppose $G$ and $H$ are points in the interior of the hexagon
such that $\angle AGB = \angle DHE = 2\pi/3$.
Prove that
\begin{align*}
AG + GB + GH + DH + HE \ge CF.
\end{align*}

\item
Let $p$ be an odd prime number.
How many $p$-element subsets $A$ of
$\{1,2,\ldots,2p\}$ are there,
the sum of whose elements is divisible by $p$?
\end{enumerate}

\newpage
\begin{center}
\textbf{37$^{\text{th}}$ International Mathematical Olympiad}\\
Mumbai, India\\
Day I \hspace{1cm} 9 a.m. -- 1:30 p.m.\\
July 10, 1996
\end{center}

\vspace{0.8cm}

\begin{enumerate}

\item
We are given a positive integer $r$ and a rectangular board $ABCD$
with dimensions $|AB|=20$, $|BC|=12$.
The rectangle is divided into a grid of $20 \times 12$ unit squares.
The following moves are permitted on the board:
one can move from one square to another only if the distance
between the centers of the two squares is $\sqrt{r}$.
The task is to find a sequence of moves leading from the square
with $A$ as a vertex to the square with $B$ as a vertex.


\begin{enumerate}
\item[(a)]
Show that the task cannot be done if $r$ is divisible by $2$ or $3$.

\item[(b)]
Prove that the task is possible when $r=73$.

\item[(c)]
Can the task be done when $r=97$?
\end{enumerate}

\item
Let $P$ be a point inside triangle $ABC$ such that
\begin{align*}
\angle APB = \angle ACB = \angle APC = \angle ABC .
\end{align*}
Let $D,E$ be the incenters of triangles $APB$, $APC$, respectively.
Show that $AP$, $BD$, $CE$ meet at a point.

\item
Let $S$ denote the set of nonnegative integers.
Find all functions $f$ from $S$ to itself such that
\begin{align*}
f(m+f(n)) = f(f(m)) + f(n)
\qquad \text{for all } m,n \in S.
\end{align*}

\end{enumerate}

\vspace{0.5cm}

\begin{center}
\textbf{37$^{\text{th}}$ International Mathematical Olympiad}\\
Mumbai, India\\
Day II \hspace{1cm} 9 a.m. -- 1:30 p.m.\\
July 11, 1996
\end{center}

\begin{enumerate}

\item
The positive integers $a$ and $b$ are such that the numbers
$15a + 16b$ and $16a - 15b$ are both squares of positive integers.
What is the least possible value that can be taken on by the smaller
of these two squares?

\item
Let $ABCDEF$ be a convex hexagon such that
$AB$ is parallel to $DE$,
$BC$ is parallel to $EF$, and
$CD$ is parallel to $FA$.
Let $R_A$, $R_C$, $R_E$ denote the circumradii of triangles
$FAB$, $BCD$, $DEF$, respectively,
and let $P$ denote the perimeter of the hexagon.
Prove that
\begin{align*}
R_A + R_C + R_E \ge \frac{P}{2}.
\end{align*}

\item
Let $p,q,n$ be three positive integers with $p + q < n$.
Let $(x_0,x_1,\ldots,x_n)$ be an $(n+1)$-tuple of integers
satisfying the following conditions:

\begin{enumerate}
\item[(a)]
$x_0 = x_n = 0$.

\item[(b)]
For each $i$ with $1 \le i \le n$,
either $x_i - x_{i-1} = p$ or
$x_i - x_{i-1} = -q$.
\end{enumerate}

Show that there exist indices $i < j$ with $(i,j) \ne (0,n)$,
such that $x_i = x_j$.

\end{enumerate}
\newpage
\begin{center}
\textbf{38$^{\text{th}}$ International Mathematical Olympiad}\\
Mar del Plata, Argentina\\
Day I\\
July 24, 1997
\end{center}

\vspace{1cm}

\begin{enumerate}

\item
In the plane the points with integer coordinates are the vertices of unit squares.
The squares are colored alternately black and white (as on a chessboard).
For any pair of positive integers $m$ and $n$, consider a right-angled triangle
whose vertices have integer coordinates and whose legs, of lengths $m$ and $n$,
lie along edges of the squares.

Let $S_1$ be the total area of the black part of the triangle and
$S_2$ be the total area of the white part. Let
\begin{align*}
f(m,n) = |S_1 - S_2|.
\end{align*}

\begin{enumerate}
\item[(a)]
Calculate $f(m,n)$ for all positive integers $m$ and $n$
which are either both even or both odd.

\item[(b)]
Prove that $f(m,n) \le \frac{1}{2}\max\{m,n\}$ for all $m$ and $n$.

\item[(c)]
Show that there is no constant $C$ such that $f(m,n) < C$ for all $m$ and $n$.
\end{enumerate}

\item
The angle at $A$ is the smallest angle of triangle $ABC$.
The points $B$ and $C$ divide the circumcircle of the triangle into two arcs.
Let $U$ be an interior point of the arc between $B$ and $C$ which does not contain $A$.
The perpendicular bisectors of $AB$ and $AC$ meet the line $AU$ at $V$ and $W$,
respectively.
The lines $BV$ and $CW$ meet at $T$.
Show that
\begin{align*}
AU = TB + TC.
\end{align*}

\item
Let $x_1,x_2,\ldots,x_n$ be real numbers satisfying the conditions
\begin{align*}
|x_1 + x_2 + \cdots + x_n| = 1
\end{align*}
\begin{align*}
and
\end{align*}
\begin{align*}
|x_i| \le \frac{n+1}{2}
\qquad i = 1,2,\ldots,n.
\end{align*}

Show that there exists a permutation $y_1,y_2,\ldots,y_n$ of
$x_1,x_2,\ldots,x_n$ such that
\begin{align*}
|y_1 + 2y_2 + \cdots + ny_n| \le \frac{n+1}{2}.
\end{align*}

\end{enumerate}
\newpage
\begin{center}
\textbf{38$^{\text{th}}$ International Mathematical Olympiad}\\
Mar del Plata, Argentina\\
Day II\\
July 25, 1997
\end{center}

\vspace{1cm}

\begin{enumerate}
\setcounter{enumi}{3}

\item
An $n \times n$ matrix whose entries come from the set
$S = \{1,2,\ldots,2n-1\}$ is called a \textit{silver matrix} if,
for each $i = 1,2,\ldots,n$, the $i$th row and the $i$th column together
contain all elements of $S$.
Show that

\begin{enumerate}
\item[(a)]
there is no silver matrix for $n = 1997$;

\item[(b)]
silver matrices exist for infinitely many values of $n$.
\end{enumerate}

\item
Find all pairs $(a,b)$ of integers $a,b \ge 1$ that satisfy the equation
\begin{align*}
a^{b^2} = b^a.
\end{align*}

\item
For each positive integer $n$, let $f(n)$ denote the number of ways of
representing $n$ as a sum of powers of $2$ with nonnegative integer
exponents.
Representations which differ only in the ordering of their summands
are considered to be the same.
For instance, $f(4) = 4$, because the number $4$ can be represented
in the following four ways:
\begin{align*}
4; 2+2; 2+1+1; 1+1+1+1.
\end{align*}

Prove that, for any integer $n \ge 3$,
\begin{align*}
2^{n^2/4} < f(2^n) < 2^{n^2/2}.
\end{align*}

\end{enumerate}

\end{document}

